\documentclass[]{article}
\usepackage[margin=0.9in]{geometry}

\usepackage{enumitem}
\usepackage{graphicx}
\usepackage{float}
\usepackage{hyperref}
\usepackage[dvipsnames]{xcolor}

% Make it easier to see when we talk about Docks and not Docker
\newcommand{\docks}{\textcolor{Blue}{Docks} }
\newcommand{\docker}{Docker }

% Images in uml/ and user-manual/ have the same names so
% uml/ has not been included here. Instead images/ is included
% so they can be referenced with uml/image.png
\graphicspath{{images/user-manual/}{images/logo/}{images/umlet/}{images/}}

%opening
\title{
	\vspace{-1.5cm}
	\includegraphics[scale=0.7]{docks_round_512.png}
	\\[1cm]
	Prize Motivation for Docks}
\author{\textbf{Team}: TripleParity\\
\textbf{Client}: Compiax\\
\\
\textbf{Team Members}\\
Francois Mentz\\
Connor Armand du Plooy\\
Raymond De Vos\\
Evert Geldenhuys\\
Anna-Marié Helberg\\
Paul Wood}

\date{}

\begin{document}

\maketitle

\tableofcontents

\pagebreak

\section{Project Prizes Motivation}
\subsection{Kindle Technologies - Fit for Purpose}

%Hierdie staan net so in die requirements dockument. Ek voel dit se wat ons moet se, maar ek het maar die paragraaf ge-add
In the beginning, we did not quite understand what the purpose of \docks should be but 
after consulting the client again and playing around with \docker the real world problem
became clear: provide a secure web user interface for using \docker.

%Dalk moet ons die meer layman se
\docker is a tool designed to make it easier to create, deploy and run applications
using lightweight virtualization. It provides a command line interface (CLI)
and RESTful API. Maintaining and deploying applications often involve multiple
people. Providing multiple people access to the \docker CLI requires
Secure Shell access (SSH) as root to the server running \docker. If the server
is secure it will only provide SSH access using public/private keys, which
reduces convenience and restricts access to devices that are SSH capable and
holds a private key. The \docker API lacks functions which are provided by
the CLI, so it cannot be used on its own.

%hierdie het ek by gesit
The requirements of the problem domain were solved by \docks. That is because we did implement
an easier way for using \docker than using the CLI. But, we also added the extra security that
was lost when we moved from the CLI to a user interface with the implementation of the two factor
authentication.

\subsection{Quant - Most innovative use of open source}
\docks utilises only open soure software. Docker is completely open source and all
of the technologies and frameworks we used are also open source. 

%Help met die asseblief
A member our out group, Evert Geldenhuys, also wrote a package that is freely 
available to the public community. \docker has the functionality where a compose file
can be used to generate a stack and services but once the stack has been deployed the 
information in the compose file cannot be extracted again. Evert wrote a package that
allows you to generate the compose file from a running stack in the swarm.


\subsection{Singular - Software Engineering Excellence}
Software Security was always our focus when we were designing and building \docks.
Our team structure was quite strange but it worked out quite well. We followed the Scrum 
framework where Evert was the scrum master. We then made a soft split between back- 
and frontend where Raymond was "sub-"scrum master for the backend and Francois "sub-"scrum
master for the frontend. The rest of the team members (Anna-Marie, Armand and Paul) 
worked mainly on frontend elements and integration with the backend. However, Raymond 
has his time working on frontend elements and Anna-Marie and Armand also worked on the 
backend. Hence, we call it a soft split and that allowed for transparency between members. 

From looking at our documentation, one can clearly understand what use purpose and use 
for \docks is. We clearly followed strict coding standards and review processes when 
developing the system. The whole group used the same IDE (Visual Studio Code) where ES Lint 
was active which means that linting errors were accounted for and easily solvable. Minimum two 
group members were required to review and approve pull requests into develop but we always
aimed to gather at least 3 if not all of the team members's feedback.

We used Telegram and Slack as our main method of communicating with each other. We planned and
managed every issue that needed to be taken care of with the help of Zenhub. Data integrity was 
ensured through the review process but also with the help of Travis which monitored every push and
every pull request. If the Travis build fails then the pull request cannot be approved. In this fashion,
we always ensured that the code in master is up to scratch and correct.


\end{document}
